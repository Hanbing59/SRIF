\documentclass[./book.tex]{subfiles}
\begin{document}

\section{正定矩阵}

\section{正定阵的属性}

\section{矩阵平方根与Cholesky分解算法}

\section{Cholesky分解的Rank One改化}
滤波算法的许多目标应用涉及如下形式的矩阵更新
\begin{equation}
    \bm{\bar{P}} = \bm{P} + c \bm{a} \bm{a}^{\textrm{T}}
\end{equation}
其中,\(\bm{P}\)为正定阵, \(c\)是标量,而\(\bm{a}\)为\(n\)维向量。

\section{观测误差的白噪声化}

\section{成对相关的观测误差}

\section{具有给定协方差的随机样本的构建}
假设要构建一组均值为零、协方差为\(\bm{P}\)的随机样本\(\bm{x}\),令\(\bm{P}=\bm{L}\bm{L}^{\textrm{T}}\),
\(\bm{y}\)为\(n\)个零均值、单位方差的独立样本构成的随机向量,则\(\bm{x}=\bm{Ly}\)具备所要求的统计属性。
此应用显然是矩阵平方根的几何解释的结果,即通过改变变量使得在新的坐标系中,向量各分量之间彼此间独立,
并通过缩放坐标轴对向量分量的方差进行单位化。

\emph{注记}:由于\(\bm{x}\)是\(\bm{y}\)的各分量的线性组合,若\(\bm{y}\)为高斯随机过程,则\(\bm{x}\)也
同样为高斯随机过程。

\section{附录}

\subsection{上三角矩阵分解算法}

\subsection{下三角Cholesky分解的FORTRAN实现}

\subsection{\texorpdfstring{UDU\textsuperscript{T}}{UDUT}更新的FORTRAN实现}

\section{参考文献}



\end{document}