\documentclass[./book.tex]{subfiles}
\begin{document}

\section{引言}
先回顾一下经典的最小二乘问题——由一组线性方程确定参数估值。

近些年,一种不同的估计方法——递推最小协方差估计(因Kalman而流行)得到广泛采用。在本章,我们将推导Kalman递推公式并讨论一种由Potter[2]提出的
精度更高、更稳定的Kalman算法平方根公式。突出这两种算法的相似性的FORTRAN实现在本章附录-4和附录-5中给出。在第5章中,将推导和讨论其他
的参数估计算法,并比较不同算法各自的优点、限制以及数值效率。

\section{线性最小二乘}

假设有线性方程系统
\begin{equation}
    \bm{z}=\bm{A}\bm{x}+\bm{v}
\end{equation}

\section{最小二乘解的统计解释}

\section{先验统计信息的引入}

\section{最小二乘信息处理器的递归}

\section{Kalman滤波数据处理}

\section{Kalman算法的Potter实现}

\section{协方差数据处理的计算考虑}

\section{附录}

\subsection{“满秩超定方程具有非奇异的法矩阵”的证明}

\subsection{一个矩阵求逆的引理}

\subsection{使用信息阵的数据处理}

\subsection{采用Kalman算法的数据处理}

\subsection{采用Potter算法的数据处理}

\section{参考文献}

\end{document}